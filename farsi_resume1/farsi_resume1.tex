% !TEX TS-program = XeLaTeX
% Commands for running this example:
% 	 xelatex Amintoosi_CV
% 	bibtex8 -W -c cp1256fa Amintoosi_CV
% 	xelatex Amintoosi_CV
% 	xelatex Amintoosi_CV
% End of Commands
\documentclass[11pt,a4paper]{bidimoderncv}
% M.Amintoosi
% برای اجرا باید دنباله کارهای زیر را انجام دهید:

\usepackage[numbers]{natbib}%
%\setlength{\bibhang}{2em}

\cvtheme[blue]{bidiclassic}%casual}
\usepackage{pifont}
%\usepackage[scale=0.8]{geometry}
\usepackage{xepersian}
\settextfont[Scale=1]{XB_Niloofar.ttf}%{B Nazanin}%
\setdigitfont{XB_Niloofar.ttf}%{B Nazanin}%
\setlatintextfont[Scale=1]{LinLibertine_RBIah.ttf}%{Linux Libertine}%{Times New Roman}%
%\defpersianfont\nastaliq[Scale=1.2]{IranNastaliq}
\defpersianfont\mytitlefont[Scale=.8]{XB_Zar.ttf}

\AtBeginDocument{\recomputelengths}
\firstname{\mytitlefont علی}
\familyname{\mytitlefont سرور‌امینی\hspace*{10mm}}
\resumename{رزومه فارسی ۱}
\address{ایران تهران}
\mobile{۰۹۳۹۵۹۹۲۴۸۶}
\email{ali.sorouramini[AT]gmail[DOT]com}
%\photo[70pt]{mAmintoosi.jpg}
\begin{document}
  \maketitle
  \vspace{-5mm}

  \section{تحصیلات}
    \cventry{۱۳۹۱-۱۳۹۵}{کارشناسی}{دانشگاه تهران}{ تهران}{}{مهندسی برق (مخابرات)}
    \cventry{۱۳۹۵-۱۳۹۸}{کارشناسی ارشد}{دانشگاه تهران}{ تهران}{}{مهندسی برق (شبکه‌های مخابراتی)}

  \section{مهارت‌ها}
    \cvlistitem{مسلط به زبان‌های برنامه نویسی C/C++ و Pthon}
    \cvlistitem{مسلط به سیستم‌عامل لینوکس}
    \cvlistitem{آشینایی با زبان‌های برنامه‌نویسی Ruby، Javascript و Lua}
    \cvlistitem{آشنایی با نرم‌افزار کنترل نسخه Git}
    \cvlistitem{آشنایی با برنامه‌نویسی هسته لینوکس}
    \cvlistitem{آشنایی با فریمورک Ruby on Rails}
    \cvlistitem{مسلط به میکروکنترلرهای ARM ساخت شرکت ST و NXP و برنامه نویسی در محیط ARM-MDK}
    \cvlistitem{آشنایی با نرم‌افزار xilinx SDK برای برنامه‌نویسی SoCهای Zynq}

  \section{پروژه‌های انجام شده}
    \cvlistitem{   مشارکت در طراحی و ساخت دستگاه تشخیص نوع فاز در سیستم‌های الکتریکی سه‌فاز به عنوان محصول دانش‌بنیان (تا سال 9۷) – پیاده سازی سرور جمع‌آوری اطلاعات اندازه‌گیری شده توسط دستگاه‌ها با استفاده از فریمورک RoR و دیپلوی توسط docker و انجام بخش‌های ارتباطی و انتقال داده‌ها به سرور با استفاده از پروتکل HTTP. نمایش داده‌های اندازه‌گیری‌شده به صورت لحظه‌ای بروی نمایشگر 4.3" لمسی رنگی. پیاده‌سازی الگوریتم تخمین پالس GPS در صورت عدم پوشش ماهواره‌ای بر روی میکرو کنترلر‌ LPC4088 در محیط برنامه نویسی MDK.}
    \cvlistitem{مشارکت در طراحی و ساخت تابلو برق هوشمند تک فاز چهل آمپر(IPDU) دارای استاندار‌های EMC طبق MIL-461 برای شرکت صنایع مخابراتی صاایران – پیاده‌سازی نمایش اطلاعات و اعمال فرامین با استفاده از نمایشگر 7" لمسی رنگی. پیاده‌سازی سرور HTTP با استفاده از زبان C++ برای کنترل تابلو  تحت شبکه برروی میکروکنترلر LPC4357 در محیط برنامه نویسی MDK. پیاده‌سازی مکانیزم ارتباطی با قابلیت اطمینان بالا بین قسمت‌های حفاظت و ارتباط تابلو (تا سال 96)}
    \cvlistitem{مشارکت در طراحی و ساخت دستگاه ثبات داده‌های محیطی تحت شبکه برای شرکت صنایع مخابراتی صاایران – انجام بخش ‌ ‌های مربوط به جمع‌آوری اطلاعات برروی شبکه و نمایش اطلاعات روی نمایشگر 7" رنگی لمسی و ذخیره اطلاعات روی حافظه داخلی. تمامی قسمت‌ها با استفاده از برد ‌ nanopi-M3ساخته شده‌است  و برای GUI از Qt 4 استفاده شده‌است(سال 95)}

    \cvlistitem{مشارکت در طراحی و ساخت دستگاه ثبات ولتاژ DC، جریان AC/DC و توان برای شرکت صنایع مخابراتی صاایران – انجام بخش مربوط به جمع‌آوری و نمایش اطلاعات روی نمایشگر 7" رنگی لمسی و ذخیره اطلاعات روی حافظه فلش با استفاده از میکروکنترلر LPC1788 در محیط برنامه نویسی MDK-ARM (سال 95)}

  \section{سایر موارد}
    \cvlistitem{رتبه ۲۳۷ منطقه ۱ کنکور سال ۱۳۹۰}
    \cvlistitem{ورود بدون کنکور به مقطع کارشناسی ارشد دانشگاه تهران رشته مهندسی برق- شبکه‌های مخابراتی}
\end{document}
