z
\usepackage{etoolbox} % Required for conditional statements

\setlength\parindent{0pt} % Stop paragraph indentation

\usepackage{supertabular} % Required for the supertabular environment which allows tables to span multiple pages so sections with tables correctly wrap across pages

%----------------------------------------------------------------------------------------
%  DOCUMENT MARGINS
%----------------------------------------------------------------------------------------

\usepackage{geometry} % Required for adjusting page dimensions and margins

\geometry{
  hmargin=1.3cm,
  vmargin=1.5cm,
  a4paper,
  %showframe
}

\usepackage{paracol} % Required for creating multi-column layouts that can span pages automatically

\columnratio{0.45,0.55} % The relative ratios of the two columns in the CV

\setlength\columnsep{0.03\textwidth} % Specify the amount of space between the columns

%----------------------------------------------------------------------------------------
%  FONTS
%----------------------------------------------------------------------------------------

\usepackage{fontspec} % Required for specifying custom fonts under XeLaTeX

\setmainfont{EBGaramond}[ % Make the default font EBGaramond
Path=fonts/, % The font is provided with the template in the fonts folder
UprightFont=*-Regular.ttf,
BoldFont=*-Bold.ttf,
BoldItalicFont=*-BoldItalic.ttf,
ItalicFont=*-Italic.ttf,
SmallCapsFont=*-SC.ttf
]

\newfontfamily\cvtextfont[Path=fonts/]{freebooterscript} % Create a new font family for the cursive font Freebooter Script, provided with the template in the fonts folder

\newfontfamily{\FA}[Path=fonts/]{FontAwesome} % Create a new font family for FontAwesome icons, provided with the template in the fonts folder
\input{fonts/fontawesomesymbols-xeluatex.tex} % Load a file to create shortcuts to the icons, see icon examples and their shortcuts in fontawesome.pdf in the fonts folder

\usepackage[sf,scale=0.95]{libertine} % Load Libertine as a \sffamily font for sans serif titles

%----------------------------------------------------------------------------------------
%  COLOURS AND LINKS
%----------------------------------------------------------------------------------------

\usepackage[usenames,svgnames]{xcolor} % Allows the definition and use of custom colours

\definecolor{text}{HTML}{2b2b2b} % Main document font colour, off-black
\definecolor{headings}{HTML}{701112} % Dark red colour for headings
\definecolor{shade}{HTML}{F5DD9D} % Peach colour for the contact information box
\definecolor{linkcolor}{HTML}{641c1d} % 25% desaturated headings colour for links
% Other colour options: shade=B9D7D9 and linkcolor=A40000; shade=D4D7FE and linkcolor=FF0080

% For preset colours that can be used by their names without having to define them, see: https://www.latextemplates.com/svgnames-colors

\color{text} % Set the default text colour for the whole document to the colour defined as 'text' above

%------------------------------------------------

\usepackage{hyperref} % Required for links

\hypersetup{colorlinks, breaklinks, urlcolor=linkcolor, linkcolor=linkcolor} % Set up links and their colours


%----------------------------------------------------------------------------------------
%  HEADERS & FOOTERS
%----------------------------------------------------------------------------------------

\usepackage{fancyhdr} % Required for customising headers and footers

\pagestyle{fancy} % Enable custom headers and footers

\fancyhf{} % This suppresses all headers and footers by default, add headers and footers in the template file as per the example

\renewcommand{\headrulewidth}{0pt} % Remove the default rule under the header

%----------------------------------------------------------------------------------------
%  SECTIONS
%----------------------------------------------------------------------------------------

\usepackage[nobottomtitles*]{titlesec} % Required for modifying sections, the nobottomtitles* is required for pushing section titles to the next page when they are close to the bottom of the page

\renewcommand{\bottomtitlespace}{0.1\textheight} % Modify the minimal space required from the bottom margin not to move the title to the next page
\newcommand*{\section}[1]{%
\vspace*{2.5ex \@plus 1ex \@minus .2ex}%
  \phantomsection{}%
  \addcontentsline{toc}{part}{#1}%
  \parbox[m]{\hintscolumnwidth}{\bidi@raggedleft\hintfont{\leavevmode\color{sectionrectanglecolor}\rule{\hintscolumnwidth}{1ex}}}%
  \hspace{\separatorcolumnwidth}%
  \parbox[m]{\maincolumnwidth}{\sectionstyle{#1}}\\[1ex]}
\newcommand*{\subsection}[1]{%
  \cvline[0.45em]{}{\subsectionstyle{#1}}}
\newcommand*{\cvline}[3][.25em]{%
  \begin{tabular}{@{}p{\hintscolumnwidth}@{\hspace{\separatorcolumnwidth}}p{\maincolumnwidth}@{}}%
    \bidi@raggedleft\hintfont{#2} &{#3}%
  \end{tabular}\\[#1]}
\newcommand*{\cvdoubleitem}[4]{%
 \cvline{#1}{\begin{minipage}[t]{\doubleitemmaincolumnwidth}#2\end{minipage}%
 \hfill%
 \begin{minipage}[t]{\hintscolumnwidth}\bidi@raggedleft\hintfont{#3}\end{minipage}\hspace*{\separatorcolumnwidth}\begin{minipage}[t]{\doubleitemmaincolumnwidth}#4\end{minipage}}}
\newcommand*{\cvlistitem}[2][\listitemsymbol{}]{%
  \cvline[0pt]{}{#1~\begin{minipage}[t]{\listitemmaincolumnwidth}#2\end{minipage}}}
\newcommand*{\cvlistdoubleitem}[3][\listitemsymbol{}]{%
  \cvline[0pt]{}{#1~\begin{minipage}[t]{\listdoubleitemmaincolumnwidth}#2\end{minipage}%
  \hfill%
  \ifthenelse{\equal{#3}{}}%
    {}%
    {#1~\begin{minipage}[t]{\listdoubleitemmaincolumnwidth}#3\end{minipage}}}}
