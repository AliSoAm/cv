\documentclass[11pt,a4paper]{bidimoderncv}
\usepackage[numbers]{natbib}%
\usepackage{tikz}
\usepackage{lipsum}
\usepackage{pifont}
\usepackage{xepersian}

\newcommand\versionAndDateText{%
  \footnotesize{
    Updated at: NA  \\
    Hash: NA
  }
}
\newcommand\printVersionAndDate{%
  \begin{tikzpicture}[remember picture,overlay]
    \node[anchor=south,yshift=0pt] at (current page.south) {\parbox{\dimexpr\textwidth-\fboxsep-\fboxrule\relax}{\versionAndDateText}};
  \end{tikzpicture}%
}
\cvtheme[blue]{bidiclassic}%casual}
\settextfont[Scale=1]{XB_Niloofar.ttf}%{B Nazanin}%
\setdigitfont{XB_Niloofar.ttf}%{B Nazanin}%
\setlatintextfont[Scale=1]{LinLibertine_R.ttf}%{Linux Libertine}%{Times New Roman}%
%\defpersianfont\nastaliq[Scale=1.2]{IranNastaliq}
\defpersianfont\mytitlefont[Scale=.8]{XB_Zar.ttf}

\AtBeginDocument{\recomputelengths}
\firstname{\mytitlefont علی}
\familyname{\mytitlefont سرور‌امینی\hspace*{10mm}}
\resumename{رزومه فارسی ۱}
\mobile{۰۹۳۹۵۹۹۲۴۸۶}
\email{ali.sorouramini@gmail.com}
\extrainfo{\httplink[https://alisoam.github.io]{alisoam.github.io}}

%\photo[70pt]{mAmintoosi.jpg}
\begin{document}
  \maketitle
  \printVersionAndDate
  \vspace{-5mm}
  \section{تحصیلات}
    \cventry{۱۳۹۱-۱۳۹۵}{کارشناسی}{دانشگاه تهران}{دانشکده مهندسی برق و کامپیوتر}{}{مهندسی برق (مخابرات) معدل ۱۸٫۲۱}
    \cventry{۱۳۹۵-۱۳۹۸}{کارشناسی ارشد}{دانشگاه تهران}{دانشکده مهندسی برق و کامپیوتر}{}{مهندسی برق (شبکه‌های مخابراتی) معدل ۱۸٫۱۵}
  \vspace{-5mm}
  \section{دروس منتخب}
    \cvline{کارشناسی}{برنامه‌ نویسی پیشرفته (۲۰)، شبکه‌های کامپیوتری (۲۰)، مدار منطقی (۲۰)، ریزپردازنده(۲۰)}
    \cvline{کارشناسی ارشد}{کلان داده (۱۹)،‌ بازشناسی الگو (۱۹)، شبکه‌های مخابرات داده (۱۹٫۷۵)}
  \vspace{-5mm}
  \section{مهارت‌ها}
    \cvlistitem{مسلط به زبان‌های برنامه نویسی \lr{C/C++} و \lr{Python}}
    \cvlistitem{مسلط به سیستم‌عامل لینوکس}
    \cvlistitem{مسلط به \lr{Docker} و \lr{OpenStack}}
    \cvlistitem{آشنایی با \lr{‌Bash Scripting}}
    \cvlistitem{آشنایی با \lr{Elastic Search} و \lr{‌Hadoop}}
    \cvlistitem{آشینایی با زبان‌های برنامه‌نویسی \lr{Ruby}، \lr{Javascript} و \lr{Lua}}
    \cvlistitem{آشنایی با \lr{HTML} و \lr{CSS}}
    \cvlistitem{آشنایی با نرم‌افزار کنترل نسخه \lr{Git}}
    \cvlistitem{آشنایی با \lr{Travis CI} و \lr{Gitlab CI}}
    \cvlistitem{آشنایی با برنامه‌نویسی هسته لینوکس}
    \cvlistitem{آشنایی با برنامه‌نویسی سوکت}
    \cvlistitem{آشنایی با فریمورک  \lr{Ruby on Rails}}
    \cvlistitem{مسلط به میکروکنترلرهای \lr{ARM} ساخت شرکت \lr{ST} و \lr{NXP} و برنامه نویسی در محیط \lr{ARM-MDK}}
    \cvlistitem{آشنایی با نرم‌افزار \lr{Xilinx SDK} برای برنامه‌نویسی \lr{SoC}های  \lr{Zynq}}
  \vspace{-5mm}
  \section{پروژه‌های انجام شده}
    \cvlistitem{   مشارکت در طراحی و ساخت دستگاه تشخیص نوع فاز در سیستم‌های الکتریکی سه‌فاز (تا تابستان 9۷)
      \begin{itemize}
        \item -
          پیاده سازی سرور جمع‌آوری اطلاعات اندازه‌گیری شده  توسط دستگاه‌ها( \href{https://phaseyab.ir}{https://phaseyab.ir}) با استفاده از فریمورک \lr{RoR} و دیپلوی توسط \lr{docker} و انجام بخش‌های ارتباطی و انتقال داده‌ها به سرور با استفاده از پروتکل \lr{HTTP}.
        \item -
          پیاده‌سازی کلاینت \lr{HTTP} مناسب برای استفاده در دستگاه‌های نهفته در زبان \lr{C++} (قابل دسترس در \href{https://github.com/alisoam/http_client}{https://github.com/alisoam/http\_client}).
        \item -
          پیاده‌سازی دستگاه‌های سیار با استفاده از میکرو کنترلر \lr{LPC1768} و دستگاه‌های مرجع با استفاده از میکروکنترلر \lr{LPC4088}
        \item -
          پیاده‌سازی الگوریتم تخمین پالس \lr{GPS} در صورت عدم پوشش ماهواره‌ای بر روی میکرو کنترلر‌های \lr{NXP}.
        \item -
          فازیابی ۷۰۰۰۰ مشترک شبکه توزیع برق در فاز پایلوت.
      \end{itemize}
    }
    \cvlistitem{مشارکت در طراحی و ساخت تابلو برق هوشمند تک فاز چهل آمپر(\lr{IPDU}) - (تا سال 96)
      \begin{itemize}
        \item -
          دارای استاندار‌های \lr{EMC} طبق \lr{MIL-461}.
        \item -
          پیاده‌سازی سرور \lr{HTTP} مناسب برای استفاده در دستگاه‌های نهفته در زبان \lr{C++} (قابل دسترس در \href{https://github.com/alisoam/http_server}{https://github.com/alisoam/http\_server}).
        \item -
          پیاده‌سازی ارتباط بین برد‌ها با استفاده از پروتکل \lr{CAN}.
        \item -
          پیاده‌سازی برد اندازه‌گیری با استفاده از میکرو کنترلر \lr{LPC4088} و برد رابط کاربری با استفاده از میکروکنترلر \lr{LPC4357}.
        \item -
          پیاده‌سازی رابط گرافیکی با استفاده از کتابخانه \lr{EMWIN}.
        \item -
          قرائت ۳۲ کانال ۴ \lr{ADC} با استفاده از یک SPI و نرخ نمونه برداری ۲۰ کیلوهرتز برای هر کانال با استفاده از \lr{DMA}.
      \end{itemize}
    }
  \vspace{-5mm}
  \section{سایر موارد}
    \cvlistitem{رتبه ۲۳۷ منطقه ۱ کنکور سال ۱۳۹۰}
    \cvlistitem{ورود بدون کنکور به مقطع کارشناسی ارشد دانشگاه تهران رشته مهندسی برق- شبکه‌های مخابراتی}
\end{document}
