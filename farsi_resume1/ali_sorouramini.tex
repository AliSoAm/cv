% !TEX TS-program = XeLaTeX
% Commands for running this example:
% 	 xelatex Amintoosi_CV
% 	bibtex8 -W -c cp1256fa Amintoosi_CV
% 	xelatex Amintoosi_CV
% 	xelatex Amintoosi_CV
% End of Commands
\documentclass[11pt,a4paper]{bidimoderncv}
% M.Amintoosi
% برای اجرا باید دنباله کارهای زیر را انجام دهید:

\usepackage[numbers]{natbib}%
%\setlength{\bibhang}{2em}

\cvtheme[blue]{bidiclassic}%casual}
\usepackage{pifont}
%\usepackage[scale=0.8]{geometry}
\usepackage{xepersian}
\settextfont[Scale=1]{XB_Niloofar.ttf}%{B Nazanin}%
\setdigitfont{XB_Niloofar.ttf}%{B Nazanin}%
\setlatintextfont[Scale=1]{LinLibertine_RBIah.ttf}%{Linux Libertine}%{Times New Roman}%
%\defpersianfont\nastaliq[Scale=1.2]{IranNastaliq}
\defpersianfont\mytitlefont[Scale=.8]{XB_Zar.ttf}

\AtBeginDocument{\recomputelengths}
\firstname{\mytitlefont علی}
\familyname{\mytitlefont سرور‌امینی\hspace*{10mm}}
\resumename{رزومه فارسی ۱}
%\address{ایران تهران}
\mobile{۰۹۳۹۵۹۹۲۴۸۶}
\email{ali.sorouramini@gmail.com}
\extrainfo{\httplink[https://AliSoAm.github.io]{AliSoAm.github.io}}


%\photo[70pt]{mAmintoosi.jpg}
\begin{document}
  \maketitle
  \vspace{-5mm}

  \section{تحصیلات}
    \cventry{۱۳۹۱-۱۳۹۵}{کارشناسی}{دانشگاه تهران}{ تهران}{}{مهندسی برق (مخابرات)}
    \cventry{۱۳۹۵-۱۳۹۸}{کارشناسی ارشد}{دانشگاه تهران}{ تهران}{}{مهندسی برق (شبکه‌های مخابراتی)}

  \section{مهارت‌ها}
    \cvlistitem{مسلط به زبان‌های برنامه نویسی C/C++ و Python}
    \cvlistitem{مسلط به سیستم‌عامل لینوکس}
    \cvlistitem{مسلط به \lr{\textrm{Docker}} و \lr{\textrm{OpenStack}}}
    \cvlistitem{آشنایی با \lr{\textrm{‌Bash Scripting}}}
    \cvlistitem{آشنایی با \lr{\textrm{‌Elastic Search}} و \lr{\textrm{‌Hadoop}}}
    \cvlistitem{آشینایی با زبان‌های برنامه‌نویسی \lr{\textrm{Ruby}}، \lr{\textrm{Javascript}} و \lr{\textrm{Lua}}}
    \cvlistitem{آشنایی با \lr{\textrm{HTML}} و \lr{\textrm{CSS}}}
    \cvlistitem{آشنایی با نرم‌افزار کنترل نسخه Git}
    \cvlistitem{آشنایی با \lr{\textrm{Travis CI}} و \lr{\textrm{Gitlab CI}}}
    \cvlistitem{آشنایی با برنامه‌نویسی هسته لینوکس}
    \cvlistitem{آشنایی با برنامه‌نویسی سوکت}
    \cvlistitem{آشنایی با فریمورک  \lr{\textrm{Ruby on Rails}}}
    \cvlistitem{مسلط به میکروکنترلرهای \lr{\textrm{ARM}} ساخت شرکت \lr{\textrm{ST}} و \lr{\textrm{NXP}} و برنامه نویسی در محیط \lr{\textrm{ARM-MDK}}}
    \cvlistitem{آشنایی با نرم‌افزار \lr{\textrm{Xilinx SDK}} برای برنامه‌نویسی \lr{\textrm{SoC}}های  \lr{\textrm{Zynq}}}

  \section{پروژه‌های انجام شده}
    \cvlistitem{   مشارکت در طراحی و ساخت دستگاه تشخیص نوع فاز در سیستم‌های الکتریکی سه‌فاز (تا تابستان 9۷)
      \begin{itemize}
        \item -
          فازیابی ۷۰۰۰۰ مشترک شبکه توزیع برق در فاز پایلوت.
        \item -
          پیاده سازی سرور جمع‌آوری اطلاعات اندازه‌گیری شده  توسط دستگاه‌ها( \href{https://phaseyab.ir}{https://phaseyab.ir}) با استفاده از فریمورک \lr{\textrm{RoR}} و دیپلوی توسط docker و انجام بخش‌های ارتباطی و انتقال داده‌ها به سرور با استفاده از پروتکل \lr{\textrm{HTTP}}.
        \item -
          پیاده‌سازی کلاینت \lr{\textrm{HTTP}} مناسب برای استفاده در دستگاه‌های نهفته در زبان \lr{\textrm{C++}} (قابل دسترس در \href{https://github.com/alisoam/http_client}{https://github.com/alisoam/http\_client}).
        \item -
          پیاده‌سازی دستگاه‌های سیار با استفاده از میکرو کنترلر \lr{\textrm{LPC1768}} و دستگاه‌های مرجع با استفاده از میکروکنترلر \lr{\textrm{LPC4088}}
        \item -
          پیاده‌سازی الگوریتم تخمین پالس \lr{\textrm{GPS}} در صورت عدم پوشش ماهواره‌ای بر روی میکرو کنترلر‌های \lr{\textrm{NXP}}.
      \end{itemize}
    }
    \cvlistitem{مشارکت در طراحی و ساخت تابلو برق هوشمند تک فاز چهل آمپر(\lr{\textrm{IPDU}}) - (تا سال 96)
      \begin{itemize}
        \item -
          دارای استاندار‌های \lr{\textrm{EMC}} طبق \lr{\textrm{MIL-461}}.
        \item -
          پیاده‌سازی سرور \lr{\textrm{HTTP}} مناسب برای استفاده در دستگاه‌های نهفته در زبان \lr{\textrm{C++}} (قابل دسترس در \href{https://github.com/alisoam/http_server}{https://github.com/alisoam/http\_server}).
        \item -
          پیاده‌سازی ارتباط بین برد‌ها با استفاده از پروتکل \lr{\textrm{CAN}}.
        \item -
          پیاده‌سازی برد اندازه‌گیری با استفاده از میکرو کنترلر \lr{\textrm{LPC4088}} و برد رابط کاربری با استفاده از میکروکنترلر \lr{\textrm{LPC4357}}.
	\item -
          پیاده‌سازی رابط گرافیکی با استفاده از کتابخانه \lr{\textrm{EMWIN}}.
        \item -
          قرائت ۳۲ کانال ۴ \lr{\textrm{ADC}} با استفاده از یک SPI و نرخ نمونه برداری 20 کیلوهرتز برای هر کانال با استفاده از \lr{\textrm{DMA}}.
      \end{itemize}
    }

  \section{سایر موارد}
    \cvlistitem{رتبه ۲۳۷ منطقه ۱ کنکور سال ۱۳۹۰}
    \cvlistitem{ورود بدون کنکور به مقطع کارشناسی ارشد دانشگاه تهران رشته مهندسی برق- شبکه‌های مخابراتی}
\end{document}
