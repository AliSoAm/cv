\documentclass[11pt,a4paper]{bidimoderncv}
\usepackage[numbers]{natbib}%
\usepackage{tikz}
\usepackage{lipsum}
\usepackage{pifont}
\usepackage{xepersian}

\newcommand\versionAndDateText{%
  \footnotesize{
  Updated at: NA \\
    Hash: NA
  }
}
\newcommand\printVersionAndDate{%
  \begin{tikzpicture}[remember picture,overlay]
    \node[anchor=south,yshift=0pt] at (current page.south) {\parbox{\dimexpr\textwidth-\fboxsep-\fboxrule\relax}{\versionAndDateText}};
  \end{tikzpicture}%
}
\cvtheme[blue]{bidiclassic}%casual}
\settextfont[Scale=1]{XB_Niloofar.ttf}%{B Nazanin}%
\setdigitfont{XB_Niloofar.ttf}%{B Nazanin}%
\setlatintextfont[Scale=1]{LinLibertine_R.ttf}%{Linux Libertine}%{Times New Roman}%
%\defpersianfont\nastaliq[Scale=1.2]{IranNastaliq}
\defpersianfont\mytitlefont[Scale=.8]{XB_Zar.ttf}

\AtBeginDocument{\recomputelengths}
\firstname{\mytitlefont علی}
\familyname{\mytitlefont سرور‌امینی\hspace*{10mm}}
\resumename{رزومه فارسی ۱}
\mobile{۰۹۳۹۵۹۹۲۴۸۶}
\email{ali.sorouramini@gmail.com}
\extrainfo{\httplink[https://AliSoAm.github.io]{AliSoAm.github.io}}

%\photo[70pt]{mAmintoosi.jpg}
\begin{document}
  \maketitle
  \vspace{-5mm}
  \printVersionAndDate

  \section{تحصیلات}
    \cventry{۱۳۹۱-۱۳۹۵}{کارشناسی}{دانشگاه تهران}{ تهران}{}{مهندسی برق (مخابرات)}
    \cventry{۱۳۹۵-۱۳۹۸}{کارشناسی ارشد}{دانشگاه تهران}{ تهران}{}{مهندسی برق (شبکه‌های مخابراتی)}

  \section{مهارت‌ها}
    \cvlistitem{مسلط به زبان‌های برنامه نویسی \lr{C/C++} و \lr{Python}}
    \cvlistitem{مسلط به زمان بند \lr{FreeRTOS}}
    \cvlistitem{مسلط به \lr{LWIP} (استک \lr{TCP/IP} مورد استفاده در دستگاه‌های نهفته)}
    \cvlistitem{مسلط به میکروکنترلرهای \lr{ARM} ساخت شرکت \lr{ST} و \lr{NXP} و برنامه نویسی در محیط \lr{ARM-MDK(Keil)}}
    \cvlistitem{مسلط به برنامه‌نویسی سوکت}
    \cvlistitem{مسلط به نرم‌افزار کنترل نسخه \lr{Git}}
    \cvlistitem{مسلط به سیستم‌عامل لینوکس}
    \cvlistitem{مسلط به \lr{Docker} و \lr{OpenStack}}
    \cvlistitem{آشنایی با \lr{Travis CI} و \lr{Gitlab CI}}
    \cvlistitem{آشنایی با نرم‌افزار \lr{Xilinx SDK} برای برنامه‌نویسی \lr{SoC}های  \lr{Zynq}}
    \cvlistitem{آشنایی با برنامه‌نویسی هسته لینوکس}
    \cvlistitem{آشنایی با \lr{Bash Scripting}}
    \cvlistitem{آشینایی با زبان‌های برنامه‌نویسی \lr{Ruby}، \lr{Javascript} و \lr{Lua}}
    \cvlistitem{آشنایی با \lr{HTML} و \lr{CSS}}
    \cvlistitem{آشنایی با فریمورک  \lr{Ruby on Rails}}
    \cvlistitem{آشنایی با \lr{‌Elastic Search} و \lr{Hadoop}}

  \section{پروژه‌های انجام شده}
    \cvlistitem{   مشارکت در طراحی و ساخت دستگاه تشخیص نوع فاز در سیستم‌های الکتریکی سه‌فاز (تا تابستان 9۷) – پیاده سازی سرور جمع‌آوری اطلاعات اندازه‌گیری شده  توسط دستگاه‌ها( \href{https://phaseyab.ir}{https://phaseyab.ir}) با استفاده از فریمورک \lr{RoR} و دیپلوی توسط \lr{docker} و انجام بخش‌های ارتباطی و انتقال داده‌ها به سرور با استفاده از پروتکل \lr{HTTP}. نمایش داده‌های اندازه‌گیری‌شده به صورت لحظه‌ای بروی نمایشگر 4.3" لمسی رنگی. پیاده‌سازی الگوریتم تخمین پالس \lr{GPS} در صورت عدم پوشش ماهواره‌ای بر روی میکرو کنترلر‌ \lr{LPC4088} در محیط برنامه نویسی \lr{MDK}.}
    \cvlistitem{مشارکت در طراحی و ساخت تابلو برق هوشمند تک فاز چهل آمپر(\lr{IPDU}) دارای استاندار‌های \lr{EMC} طبق \lr{MIL-461} برای شرکت صنایع مخابراتی صاایران – پیاده‌سازی نمایش اطلاعات و اعمال فرامین با استفاده از نمایشگر 7" لمسی رنگی. پیاده‌سازی سرور \lr{HTTP} با استفاده از زبان C++ برای کنترل تابلو  تحت شبکه برروی میکروکنترلر \lr{LPC4357} در محیط برنامه نویسی \lr{MDK}. پیاده‌سازی مکانیزم ارتباطی با قابلیت اطمینان بالا بین قسمت‌های حفاظت و ارتباط تابلو (تا سال 96)}
    \cvlistitem{مشارکت در طراحی و ساخت دستگاه ثبات داده‌های محیطی تحت شبکه برای شرکت صنایع مخابراتی صاایران – انجام بخش ‌ ‌های مربوط به جمع‌آوری اطلاعات برروی شبکه و نمایش اطلاعات روی نمایشگر 7" رنگی لمسی و ذخیره اطلاعات روی حافظه داخلی. تمامی قسمت‌ها با استفاده از برد \lr{nanopi-M3} ساخته شده‌است  و برای \lr{GUI} از \lr{Qt 4} استفاده شده‌است(سال 95)}

    \cvlistitem{مشارکت در طراحی و ساخت دستگاه ثبات ولتاژ DC، جریان \lr{AC/DC} و توان برای شرکت صنایع مخابراتی صاایران – انجام بخش مربوط به جمع‌آوری و نمایش اطلاعات روی نمایشگر 7" رنگی لمسی و ذخیره اطلاعات روی حافظه فلش با استفاده از میکروکنترلر \lr{LPC1788} در محیط برنامه نویسی \lr{MDK-ARM} (سال 95)}

  \section{سایر موارد}
    \cvlistitem{رتبه ۲۳۷ منطقه ۱ کنکور سال ۱۳۹۰}
    \cvlistitem{ورود بدون کنکور به مقطع کارشناسی ارشد دانشگاه تهران رشته مهندسی برق- شبکه‌های مخابراتی}
\end{document}
