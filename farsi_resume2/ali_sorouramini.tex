\documentclass[11pt,a4paper]{bidimoderncv}
\usepackage[numbers]{natbib}%
\usepackage{tikz}
\usepackage{lipsum}
\usepackage{pifont}
\usepackage{xepersian}

\newcommand\versionAndDateText{%
  \footnotesize{
  Updated at: NA \\
    Hash: NA
  }
}
\newcommand\printVersionAndDate{%
  \begin{tikzpicture}[remember picture,overlay]
    \node[anchor=south,yshift=0pt] at (current page.south) {\parbox{\dimexpr\textwidth-\fboxsep-\fboxrule\relax}{\versionAndDateText}};
  \end{tikzpicture}%
}
\cvtheme[blue]{bidiclassic}%casual}
\settextfont[Scale=1]{XB_Niloofar.ttf}%{B Nazanin}%
\setdigitfont{XB_Niloofar.ttf}%{B Nazanin}%
\setlatintextfont[Scale=1]{LinLibertine_RBIah.ttf}%{Linux Libertine}%{Times New Roman}%
%\defpersianfont\nastaliq[Scale=1.2]{IranNastaliq}
\defpersianfont\mytitlefont[Scale=.8]{XB_Zar.ttf}

\AtBeginDocument{\recomputelengths}
\firstname{\mytitlefont علی}
\familyname{\mytitlefont سرور‌امینی\hspace*{10mm}}
\resumename{رزومه فارسی ۱}
\mobile{۰۹۳۹۵۹۹۲۴۸۶}
\email{ali.sorouramini@gmail.com}
\extrainfo{\httplink[https://AliSoAm.github.io]{AliSoAm.github.io}}

%\photo[70pt]{mAmintoosi.jpg}
\begin{document}
  \maketitle
  \vspace{-5mm}
  \printVersionAndDate

  \section{تحصیلات}
    \cventry{۱۳۹۱-۱۳۹۵}{کارشناسی}{دانشگاه تهران}{ تهران}{}{مهندسی برق (مخابرات)}
    \cventry{۱۳۹۵-۱۳۹۸}{کارشناسی ارشد}{دانشگاه تهران}{ تهران}{}{مهندسی برق (شبکه‌های مخابراتی)}

  \section{مهارت‌ها}
    \cvlistitem{مسلط به زبان‌های برنامه نویسی C/C++ و Python}
    \cvlistitem{مسلط به زمان بند \lr{\textrm{FreeRTOS}}}
    \cvlistitem{مسلط به \lr{\textrm{LWIP}} (استک \lr{\textrm{TCP/IP}} مورد استفاده در دستگاه‌های نهفته)}
    \cvlistitem{مسلط به میکروکنترلرهای \lr{\textrm{ARM}} ساخت شرکت \lr{\textrm{ST}} و \lr{\textrm{NXP}} و برنامه نویسی در محیط \lr{\textrm{ARM-MDK(Keil)}}}
    \cvlistitem{مسلط به برنامه‌نویسی سوکت}
    \cvlistitem{مسلط به نرم‌افزار کنترل نسخه \lr{\textrm{Git}}}
    \cvlistitem{مسلط به سیستم‌عامل لینوکس}
    \cvlistitem{مسلط به \lr{\textrm{Docker}} و \lr{\textrm{OpenStack}}}
    \cvlistitem{آشنایی با \lr{\textrm{Travis CI}} و \lr{\textrm{Gitlab CI}}}
    \cvlistitem{آشنایی با نرم‌افزار \lr{\textrm{Xilinx SDK}} برای برنامه‌نویسی \lr{\textrm{SoC}}های  \lr{\textrm{Zynq}}}
    \cvlistitem{آشنایی با برنامه‌نویسی هسته لینوکس}
    \cvlistitem{آشنایی با \lr{\textrm{‌Bash Scripting}}}
    \cvlistitem{آشینایی با زبان‌های برنامه‌نویسی \lr{\textrm{Ruby}}، \lr{\textrm{Javascript}} و \lr{\textrm{Lua}}}
    \cvlistitem{آشنایی با \lr{\textrm{HTML}} و \lr{\textrm{CSS}}}
    \cvlistitem{آشنایی با فریمورک  \lr{\textrm{Ruby on Rails}}}
    \cvlistitem{آشنایی با \lr{\textrm{‌Elastic Search}} و \lr{\textrm{‌Hadoop}}}

  \section{پروژه‌های انجام شده}
    \cvlistitem{   مشارکت در طراحی و ساخت دستگاه تشخیص نوع فاز در سیستم‌های الکتریکی سه‌فاز (تا تابستان 9۷) – پیاده سازی سرور جمع‌آوری اطلاعات اندازه‌گیری شده  توسط دستگاه‌ها( \href{https://phaseyab.ir}{https://phaseyab.ir}) با استفاده از فریمورک \lr{\textrm{RoR}} و دیپلوی توسط docker و انجام بخش‌های ارتباطی و انتقال داده‌ها به سرور با استفاده از پروتکل \lr{\textrm{HTTP}}. نمایش داده‌های اندازه‌گیری‌شده به صورت لحظه‌ای بروی نمایشگر 4.3" لمسی رنگی. پیاده‌سازی الگوریتم تخمین پالس \lr{\textrm{GPS}} در صورت عدم پوشش ماهواره‌ای بر روی میکرو کنترلر‌ \lr{\textrm{LPC4088}} در محیط برنامه نویسی \lr{\textrm{MDK}}.}
    \cvlistitem{مشارکت در طراحی و ساخت تابلو برق هوشمند تک فاز چهل آمپر(\lr{\textrm{IPDU}}) دارای استاندار‌های \lr{\textrm{EMC}} طبق \lr{\textrm{MIL-461}} برای شرکت صنایع مخابراتی صاایران – پیاده‌سازی نمایش اطلاعات و اعمال فرامین با استفاده از نمایشگر 7" لمسی رنگی. پیاده‌سازی سرور \lr{\textrm{HTTP}} با استفاده از زبان C++ برای کنترل تابلو  تحت شبکه برروی میکروکنترلر \lr{\textrm{LPC4357}} در محیط برنامه نویسی \lr{\textrm{MDK}}. پیاده‌سازی مکانیزم ارتباطی با قابلیت اطمینان بالا بین قسمت‌های حفاظت و ارتباط تابلو (تا سال 96)}
    \cvlistitem{مشارکت در طراحی و ساخت دستگاه ثبات داده‌های محیطی تحت شبکه برای شرکت صنایع مخابراتی صاایران – انجام بخش ‌ ‌های مربوط به جمع‌آوری اطلاعات برروی شبکه و نمایش اطلاعات روی نمایشگر 7" رنگی لمسی و ذخیره اطلاعات روی حافظه داخلی. تمامی قسمت‌ها با استفاده از برد \lr{\textrm{nanopi-M3}} ساخته شده‌است  و برای \lr{\textrm{GUI}} از \lr{\textrm{Qt 4}} استفاده شده‌است(سال 95)}

    \cvlistitem{مشارکت در طراحی و ساخت دستگاه ثبات ولتاژ DC، جریان \lr{\textrm{AC/DC}} و توان برای شرکت صنایع مخابراتی صاایران – انجام بخش مربوط به جمع‌آوری و نمایش اطلاعات روی نمایشگر 7" رنگی لمسی و ذخیره اطلاعات روی حافظه فلش با استفاده از میکروکنترلر \lr{\textrm{LPC1788}} در محیط برنامه نویسی \lr{\textrm{MDK-ARM}} (سال 95)}

  \section{سایر موارد}
    \cvlistitem{رتبه ۲۳۷ منطقه ۱ کنکور سال ۱۳۹۰}
    \cvlistitem{ورود بدون کنکور به مقطع کارشناسی ارشد دانشگاه تهران رشته مهندسی برق- شبکه‌های مخابراتی}
\end{document}
